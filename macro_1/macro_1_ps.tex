\documentclass[10pt, a4paper]{extarticle}
\setlength{\parskip}{0.5em}
%%% Работа с русским языком
\usepackage{cmap}					% поиск в PDF
\usepackage{mathtext} 				% русские буквы в формулах
\usepackage[T2A]{fontenc}			% кодировка
\usepackage[utf8]{inputenc}			% кодировка исходного текста
\usepackage[english,russian]{babel}	% локализация и переносы
\usepackage{mathtools}   % loads »amsmath«
\usepackage{graphicx}
\usepackage{caption}
\usepackage{physics}
\usepackage{subcaption}
\usepackage{tikz}
\usepackage{multicol}
\usepackage{enumitem}

\usepackage{hyperref}
\hypersetup{
colorlinks=true,
linkcolor=magenta
}

%%% Дополнительная работа с математикой
\usepackage{amsmath,amsfonts,amssymb,amsthm,mathtools} % AMS
\usepackage{icomma} % "Умная" запятая: $0,2$ --- число, $0, 2$ --- перечисление

%% Шрифты
\usepackage{euscript}	 % Шрифт Евклид
\usepackage{mathrsfs} % Красивый матшрифт

\title{Подборка \\ «Фискальная и монетарная политика. Экономический рост. Деловые циклы»}
\author{Факультатив «Качественные задачи по экономике» \\ Лицей НИУ ВШЭ}

\usepackage{geometry}
\geometry{
	a4paper,
	left=20mm,
	top=20mm,
	right=20mm
}
\setlength{\parindent}{0cm}

\DeclareMathOperator{\Lin}{\mathrm{Lin}}
\DeclareMathOperator{\Linp}{\Lin^{\perp}}
\DeclareMathOperator*\plim{plim}
%\DeclareMathOperator{\grad}{grad}
\DeclareMathOperator{\card}{card}
\DeclareMathOperator{\sgn}{sign}
\DeclareMathOperator{\sign}{sign}

\DeclareMathOperator*{\argmin}{arg\,min}
\DeclareMathOperator*{\argmax}{arg\,max}
\DeclareMathOperator*{\amn}{arg\,min}
\DeclareMathOperator*{\amx}{arg\,max}
\DeclareMathOperator{\cov}{Cov}
\DeclareMathOperator{\Var}{Var}
\DeclareMathOperator{\Cov}{Cov}
\DeclareMathOperator{\Corr}{Corr}
\DeclareMathOperator{\pCorr}{pCorr}
\DeclareMathOperator{\E}{\mathbb{E}}
\let\P\relax
\DeclareMathOperator{\P}{\mathbb{P}}



\newcommand{\cN}{\mathcal{N}}
\newcommand{\cU}{\mathcal{U}}
\newcommand{\cBinom}{\mathcal{Binom}}
\newcommand{\cPois}{\mathcal{Pois}}
\newcommand{\cBeta}{\mathcal{Beta}}
\newcommand{\cGamma}{\mathcal{Gamma}}

\def \R{\mathbb{R}}
\def \N{\mathbb{N}}
\def \Z{\mathbb{Z}}





\newcommand{\dx}[1]{\,\mathrm{d}#1} % для интеграла: маленький отступ и прямая d
\newcommand{\ind}[1]{\mathbbm{1}_{\{#1\}}} % Индикатор события
%\renewcommand{\to}{\rightarrow}
\newcommand{\eqdef}{\mathrel{\stackrel{\rm def}=}}
\newcommand{\iid}{\mathrel{\stackrel{\rm i.\,i.\,d.}\sim}}
\newcommand{\const}{\mathrm{const}}


% вместо горизонтальной делаем косую черточку в нестрогих неравенствах
\renewcommand{\le}{\leqslant}
\renewcommand{\ge}{\geqslant}
\renewcommand{\leq}{\leqslant}
\renewcommand{\geq}{\geqslant}

\renewcommand{\rmdefault}{cmss}
%\renewcommand{\ttdefault}{cmss}
\usepackage{sfmath}

\usepackage{enumitem}

\begin{document}

\maketitle

\section{Экономический рост}
\begin{enumerate}
	\item Что такое экономический рост? 
	\item Зачем нужен экономический рост?
	\item Как измерить экономический рост? 
	\item Как изменяется структура ВВП по мере развития экономики? 
	\item Приведите проблемы ВВП как индикатора благосостояния.
	\item Почему сложно посчитать ВВП Советского Союза? 
	\item Чем отличаются номинальный и реальный ВВП?
	\item Как оценить темпы экономического роста? 
	\item Что такое ППС? Что такое эффект Балассы-Самуэльсона? 
	\item Что такое инфляция? Что такое дефляция?
	\item Как инфляция влияет на экономический рост? 
	\item Как гиперинфляция влияет на экономический рост? 
	\item Как статистически связаны безработица и темпы роста ВВП?
	\item Является ли накопление капитала необходимым условием экономического роста? А достаточным?
	\item Что является основным источником экономического роста по современным представлениям?
	\item Объясните термин «Большая дивергенция». 
\end{enumerate}

\section{Деловые циклы}
\begin{enumerate}
	\item Совпадает ли ежегодный темп прироста ВВП с темпом долгосрочного роста ВВП? 
	\item Чем плохи рецессии и депрессии?
	\item Имеют ли деловые циклы фиксированный период и/или амплитуду колебаний?
	\item Симметричны ли деловые циклы (равны ли по длине рецессии и оживления)? 
	\item Меняются ли деловые циклы со временем? 
	\item Совпадает ли динамика компонентов ВВП с динамикой самого ВВП?
	\item Какой компонент ВВП наиболее волатилен?
	\item Что такое контрциклические, проциклические и антициклические показатели?
	\item Что такое опережающие, запаздывающие и совпадающие показатели?
	\item Связаны ли деловые циклы с изменениями на рынке труда? 
	\item В каких странах деловые колебания происходят чаще: в богатых или бедных? 
\end{enumerate}

\section{Фискальная и монетарная политика}
\begin{enumerate}
	\item Кто осуществляет фискальную и монетарную политику?
	\item Каковы инструменты фискальной и монетарной политики?
	\item Каковы микроэкономические цели фискальной политики?
	\item Каковы макроэкономические цели фискальной политики?
	\item Каковы цели монетарной политики? 
	\item Каковы основные ограничения макроэкономической политики?
\end{enumerate}

\section{Взаимодействие фискальной и монетарной политики}
\begin{enumerate}
	\item Какими способами государство может финансировать бюджетный дефицит? Какие проблемы с этим связаны?
\end{enumerate}

\section{Монетарная политика}
\begin{enumerate}
	\item Что такое монетарная экономика?
	\item С точки зрения монетарной экономики, когда «заканчивается» микроэкономика и начинается «макроэкономика»?
	\item Каковы микро- и макроэкономические вопросы монетарной экономики?
	\item Что такое деньги?
	\item Какими функциями обладают деньги как финансовый актив? 
	\item Каковы функции денег?
	\item Что такое бартер? Каковы его недостатки?
	\item Что такое товарные и символические деньги?
\end{enumerate}

\section{Фискальная политика}
\begin{enumerate}
	\item Почему фискальная политика часто приводит к дефициту госбюджета? 
\end{enumerate}

\end{document}