\documentclass[10pt, a4paper]{extarticle}
\setlength{\parskip}{0.5em}
%%% Работа с русским языком
\usepackage{cmap}					% поиск в PDF
\usepackage{mathtext} 				% русские буквы в формулах
\usepackage[T2A]{fontenc}			% кодировка
\usepackage[utf8]{inputenc}			% кодировка исходного текста
\usepackage[english,russian]{babel}	% локализация и переносы
\usepackage{mathtools}   % loads »amsmath«
\usepackage{graphicx}
\usepackage{caption}
\usepackage{physics}
\usepackage{subcaption}
\usepackage{tikz}
\usepackage{multicol}
\usepackage{enumitem}

\usepackage{hyperref}
\hypersetup{
colorlinks=true,
linkcolor=magenta
}

%%% Дополнительная работа с математикой
\usepackage{amsmath,amsfonts,amssymb,amsthm,mathtools} % AMS
\usepackage{icomma} % "Умная" запятая: $0,2$ --- число, $0, 2$ --- перечисление

%% Шрифты
\usepackage{euscript}	 % Шрифт Евклид
\usepackage{mathrsfs} % Красивый матшрифт

\title{Подборка \\ «Валютные кризисы. Регулирование банковского сектора»}
\author{Факультатив «Качественные задачи по экономике» \\ Лицей НИУ ВШЭ}

\usepackage{geometry}
\geometry{
	a4paper,
	left=20mm,
	top=20mm,
	right=20mm
}
\setlength{\parindent}{0cm}
\usepackage{array}
\newcommand{\PreserveBackslash}[1]{\let\temp=\\#1\let\\=\temp}
\newcolumntype{C}[1]{>{\PreserveBackslash\centering}p{#1}}

\DeclareMathOperator{\Lin}{\mathrm{Lin}}
\DeclareMathOperator{\Linp}{\Lin^{\perp}}
\DeclareMathOperator*\plim{plim}
%\DeclareMathOperator{\grad}{grad}
\DeclareMathOperator{\card}{card}
\DeclareMathOperator{\sgn}{sign}
\DeclareMathOperator{\sign}{sign}

\DeclareMathOperator*{\argmin}{arg\,min}
\DeclareMathOperator*{\argmax}{arg\,max}
\DeclareMathOperator*{\amn}{arg\,min}
\DeclareMathOperator*{\amx}{arg\,max}
\DeclareMathOperator{\cov}{Cov}
\DeclareMathOperator{\Var}{Var}
\DeclareMathOperator{\Cov}{Cov}
\DeclareMathOperator{\Corr}{Corr}
\DeclareMathOperator{\pCorr}{pCorr}
\DeclareMathOperator{\E}{\mathbb{E}}
\let\P\relax
\DeclareMathOperator{\P}{\mathbb{P}}



\newcommand{\cN}{\mathcal{N}}
\newcommand{\cU}{\mathcal{U}}
\newcommand{\cBinom}{\mathcal{Binom}}
\newcommand{\cPois}{\mathcal{Pois}}
\newcommand{\cBeta}{\mathcal{Beta}}
\newcommand{\cGamma}{\mathcal{Gamma}}

\def \R{\mathbb{R}}
\def \N{\mathbb{N}}
\def \Z{\mathbb{Z}}





\newcommand{\dx}[1]{\,\mathrm{d}#1} % для интеграла: маленький отступ и прямая d
\newcommand{\ind}[1]{\mathbbm{1}_{\{#1\}}} % Индикатор события
%\renewcommand{\to}{\rightarrow}
\newcommand{\eqdef}{\mathrel{\stackrel{\rm def}=}}
\newcommand{\iid}{\mathrel{\stackrel{\rm i.\,i.\,d.}\sim}}
\newcommand{\const}{\mathrm{const}}


% вместо горизонтальной делаем косую черточку в нестрогих неравенствах
\renewcommand{\le}{\leqslant}
\renewcommand{\ge}{\geqslant}
\renewcommand{\leq}{\leqslant}
\renewcommand{\geq}{\geqslant}

\renewcommand{\rmdefault}{cmss}
%\renewcommand{\ttdefault}{cmss}
\usepackage{sfmath}

\usepackage{enumitem}

\begin{document}

\maketitle

\section{Валютные кризисы}
Заполните таблицу:

\def\arraystretch{3}
\begin{center}
\begin{tabular}{l|C{4cm}|C{4cm}|C{4cm}}
	&I поколение & II поколение & III поколение \\[1.8ex]
	\hline
	Причина кризиса &&&  \\ 
	Механизм кризиса &&& \\
	Критика модели &&& \\
	Примеры &&&
\end{tabular}
\end{center}

\section{Регулирование банковского сектора}
\begin{enumerate}[label=\alph*)]
	\item Какие цели преследует регулирование банковского сектора?
	\item Какими способами ЦБ может регулировать банковский сектор.
	\item Чем банковская паника отличается от банковского набега?
	\item Поясните, почему банковская паника представляет особую угрозу банковскому сектору.
	\item Поясните, почему банковская паника может возникать даже в отсутствии фундаментальных причин.
	\item Поясните, как страхование вкладов может решить проблему банковской паники из-за влияния ожиданий агентов.
	\item Объясните, какие функции выполняет АСВ.
	\item Поясните, что такое санация банков.
	\item Назовите примеры причин, при которых у коммерческого банка может быть отозвана лицензия.
	\item Назовите побочные эффекты регулирования банковского сектора.
	\item Что такое Базель I, II и III?
\end{enumerate}


\end{document}