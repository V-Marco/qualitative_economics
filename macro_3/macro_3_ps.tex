\documentclass[10pt, a4paper]{extarticle}
\setlength{\parskip}{0.5em}
%%% Работа с русским языком
\usepackage{cmap}					% поиск в PDF
\usepackage{mathtext} 				% русские буквы в формулах
\usepackage[T2A]{fontenc}			% кодировка
\usepackage[utf8]{inputenc}			% кодировка исходного текста
\usepackage[english,russian]{babel}	% локализация и переносы
\usepackage{mathtools}   % loads »amsmath«
\usepackage{graphicx}
\usepackage{caption}
\usepackage{physics}
\usepackage{subcaption}
\usepackage{tikz}
\usepackage{multicol}
\usepackage{enumitem}

\usepackage{hyperref}
\hypersetup{
colorlinks=true,
linkcolor=magenta
}

%%% Дополнительная работа с математикой
\usepackage{amsmath,amsfonts,amssymb,amsthm,mathtools} % AMS
\usepackage{icomma} % "Умная" запятая: $0,2$ --- число, $0, 2$ --- перечисление

%% Шрифты
\usepackage{euscript}	 % Шрифт Евклид
\usepackage{mathrsfs} % Красивый матшрифт

\title{Подборка \\ «Динамическая несогласованность. Режимы монетарной политики. Особенности макроэкономической политики развивающихся стран»}
\author{Факультатив «Качественные задачи по экономике» \\ Лицей НИУ ВШЭ}

\usepackage{geometry}
\geometry{
	a4paper,
	left=20mm,
	top=20mm,
	right=20mm
}
\setlength{\parindent}{0cm}
\usepackage{array}
\newcommand{\PreserveBackslash}[1]{\let\temp=\\#1\let\\=\temp}
\newcolumntype{C}[1]{>{\PreserveBackslash\centering}p{#1}}

\DeclareMathOperator{\Lin}{\mathrm{Lin}}
\DeclareMathOperator{\Linp}{\Lin^{\perp}}
\DeclareMathOperator*\plim{plim}
%\DeclareMathOperator{\grad}{grad}
\DeclareMathOperator{\card}{card}
\DeclareMathOperator{\sgn}{sign}
\DeclareMathOperator{\sign}{sign}

\DeclareMathOperator*{\argmin}{arg\,min}
\DeclareMathOperator*{\argmax}{arg\,max}
\DeclareMathOperator*{\amn}{arg\,min}
\DeclareMathOperator*{\amx}{arg\,max}
\DeclareMathOperator{\cov}{Cov}
\DeclareMathOperator{\Var}{Var}
\DeclareMathOperator{\Cov}{Cov}
\DeclareMathOperator{\Corr}{Corr}
\DeclareMathOperator{\pCorr}{pCorr}
\DeclareMathOperator{\E}{\mathbb{E}}
\let\P\relax
\DeclareMathOperator{\P}{\mathbb{P}}



\newcommand{\cN}{\mathcal{N}}
\newcommand{\cU}{\mathcal{U}}
\newcommand{\cBinom}{\mathcal{Binom}}
\newcommand{\cPois}{\mathcal{Pois}}
\newcommand{\cBeta}{\mathcal{Beta}}
\newcommand{\cGamma}{\mathcal{Gamma}}

\def \R{\mathbb{R}}
\def \N{\mathbb{N}}
\def \Z{\mathbb{Z}}





\newcommand{\dx}[1]{\,\mathrm{d}#1} % для интеграла: маленький отступ и прямая d
\newcommand{\ind}[1]{\mathbbm{1}_{\{#1\}}} % Индикатор события
%\renewcommand{\to}{\rightarrow}
\newcommand{\eqdef}{\mathrel{\stackrel{\rm def}=}}
\newcommand{\iid}{\mathrel{\stackrel{\rm i.\,i.\,d.}\sim}}
\newcommand{\const}{\mathrm{const}}


% вместо горизонтальной делаем косую черточку в нестрогих неравенствах
\renewcommand{\le}{\leqslant}
\renewcommand{\ge}{\geqslant}
\renewcommand{\leq}{\leqslant}
\renewcommand{\geq}{\geqslant}

\renewcommand{\rmdefault}{cmss}
%\renewcommand{\ttdefault}{cmss}
\usepackage{sfmath}

\usepackage{enumitem}

\begin{document}

\maketitle

\section{Динамическая несогласованность}
\begin{enumerate}[label=\alph*)]
	\item Как Вы считаете, имеет ли сеньораж большое значение в развитых странах?
	\item Объясните интуитивно, почему в краткосрочном периоде существует выбор между инфляцией и безработицей. Существует ли этот выбор в долгосрочном периоде и почему? Является ли этот выбор осознанным?
	\item Почему политики могут быть не готовы снизить инфляцию, даже если она кажется им слишком высокой?
	\item Поясните идею Кидлэнда и Прескотта о динамически несогласованной политике.
	\item Что такое оптимальная политика? 
	\item Что такое динамически согласованная политика?
	\item Как можно решить проблему динамической несогласованности при помощи политики правил? 
	\item Какова роль репутации при решении проблемы динамической несогласованности? 
	\item Как делегирование полномочий и независимость ЦБ связаны с решением проблемы динамической несогласованности?
\end{enumerate}

\section{Режимы монетарной политики}
\begin{enumerate}[label=\alph*)]
	\item Перечислите основные режимы монетарной политики.
	\item В чём преимущества и недостатки дискреционной (гибкой) монетарной политики и политики правил?
	\item Объясните, какие инструменты входят в политику таргетирования.
	\item Перечислите режимы валютного курса. В чём их преимущества и недостатки?
	\item Что необходимо для осуществления успешной политики таргетирования денежной массы? В чём преимущества и недостатки этого вида таргетирования?
	\item Что характерно для инфляционного таргетирования? Что такое строгое и гибкое ИТ?
\end{enumerate}

\section{Особенности макроэкономической политики развивающихся стран}
\begin{enumerate}[label=\alph*)]
	\item В чём состоят особенности финансовой системы развивающихся стран?
	\item Что такое долларизация? К каким негативным последствиям она может привести? 
	\item В чём причины высокой инфляции в развивающихся странах?
\end{enumerate}



\end{document}